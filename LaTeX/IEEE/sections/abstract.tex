% !TEX root = ../main.tex

\begin{abstract}

We revisit the 1997 PayWord credit-based micropayment scheme from Rivest and Shamir. We observe that smart contracts can be used to augment ... 

%Every blockchain transaction is verified and stored on the network nodes which maintain this decentralized system. For that, such systems sooner than later are expected to face storage scalability issues which may ultimately turn them into data center-based centralized ones. Additionally, the adoption of blockchain payments for everyday purchases is not widely convenient yet. Mainly, because a transaction requires a substantial amount of time to get confirmed on the blockchain so that a merchant is confident to get credited for the consumed goods. Accordingly, there is a strong direction to come up with cryptocurrency payment mechanisms that enable offchain irrefutable transactions which can be aggregated to fewer onchain committed ones. In this work, we propose an Ethereum-based smart contract named \EthWords which mediates the process of conducting offchain and completely offline micropayments between two distrustful parties. More precisely, using PayWord hash chains, two parties engage in an \EthWords smart contract which locks the exhaustion of a given chain to only one of those parties, thus enabling only such party to verify micropayments offline, and finally successfully commit aggregated transactions onchain. While other proposals allow more than two parties to get credited from one account, a continuous online monitoring of the blockchain is required in order to catch dishonest parties. However, in \EthWords, no blockchain monitoring is required as a contract credits only one account.    

\end{abstract}

\begin{IEEEkeywords}
Cryptocurrency; Bitcoin; Blockchain; Ethereum; Smart Contracts; Payment Channels
\end{IEEEkeywords}
